\documentclass[preview]{standalone}

\usepackage[english]{babel}
\usepackage[utf8]{inputenc}
\usepackage[T1]{fontenc}
\usepackage{lmodern}
\usepackage{amsmath}
\usepackage{amssymb}
\usepackage{dsfont}
\usepackage{setspace}
\usepackage{tipa}
\usepackage{relsize}
\usepackage{textcomp}
\usepackage{mathrsfs}
\usepackage{calligra}
\usepackage{wasysym}
\usepackage{ragged2e}
\usepackage{physics}
\usepackage{xcolor}
\usepackage{microtype}
\DisableLigatures{encoding = *, family = * }
\linespread{1}

\begin{document}

A robot is mechanically constructed by connecting a set of bodies, called links,to each other using various types of joints.\\Actuators, such as electric motors, deliver forces or torques that cause the robot’s links to move.\\Usually an end-effector, such as a gripper or hand for grasping and manipulating objects, is attached to a specific link.\\All the robots considered in this course have links that can be modeled as rigid bodies\\Complete all the assignments\\Ask anything anytime!!!\\

\end{document}
